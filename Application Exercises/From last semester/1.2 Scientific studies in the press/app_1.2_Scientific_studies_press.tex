\documentclass[11pt]{article}
\usepackage[top=3cm,bottom=3cm,left=3cm,right= 3cm]{geometry}
%\geometry{landscape}                % Activate for for rotated page geometry
\usepackage[parfill]{parskip}    % Activate to begin paragraphs with an empty line rather than an indent
\usepackage{graphicx}
\usepackage{amssymb}
\usepackage{epstopdf}
\usepackage{amsmath}            
\usepackage{multirow}    
\usepackage{changepage}
\usepackage{lscape}
\usepackage{ulem}
\usepackage{multicol}
\usepackage{dashrule}
\usepackage[usenames,dvipsnames]{color}       
\usepackage{enumerate}
\newcommand{\urlwofont}[1]{\urlstyle{same}\url{#1}}
\newcommand{\degree}{\ensuremath{^\circ}}

\DeclareGraphicsRule{.tif}{png}{.png}{`convert #1 `dirname #1`/`basename #1 .tif`.png}

\newenvironment{choices}{
\begin{enumerate}[(a)]
}{\end{enumerate}}

%\newcommand{\soln}[1]{\textcolor{MidnightBlue}{\textit{#1}}}	% delete #1 to get rid of solutions for handouts
\newcommand{\soln}[1]{ \vspace{2.7cm} }

\newcommand{\solnMult}[1]{\textbf{\textcolor{MidnightBlue}{\textit{#1}}}}	% uncomment for solutions
%\newcommand{\solnMult}[1]{ #1 }	% uncomment for handouts

%\newcommand{\pts}[1]{ \textbf{{\footnotesize \textcolor{black}{(#1)}}} }	% uncomment for handouts
\newcommand{\pts}[1]{ \textbf{{\footnotesize \textcolor{red}{(#1)}}} }	% uncomment for handouts

\newcommand{\note}[1]{ \textbf{\textcolor{red}{[#1]}} }	% uncomment for handouts

\definecolor{oiG}{rgb}{.298,.447,.114}
\definecolor{oiB}{rgb}{.337,.608,.741}

\usepackage[colorlinks=false,pdfborder={0 0 0},urlcolor= oiG,colorlinks=true,linkcolor= oiG, citecolor= oiG,backref=true]{hyperref}

%\usepackage{draftwatermark}
%\SetWatermarkScale{4}

\thispagestyle{empty}

\usepackage{titlesec}
\titleformat{\section}
{\color{oiB}\normalfont\Large\bfseries}
{\color{oiB}\thesection}{1em}{}
\titleformat{\subsection}
{\color{oiB}\normalfont}
{\color{oiB}\thesubsection}{1em}{}

\newcommand{\ttl}[1]{ \textsc{{\LARGE \textbf{{\color{oiB} #1} } }}}

\newcommand{\tl}[1]{ \textsc{{\large \textbf{{\color{oiB} #1} } }}}

\begin{document}

Dr. \c{C}etinkaya-Rundel \hfill Data Analysis and Statistical Inference \\

\ttl{Application exercise: \\
1.2 Scientific studies in the press} \\

You have \textcolor{oiB}{{\sc 12 minutes}} to complete this application exercise. Submit your responses on \href{https://sakai.duke.edu/portal/site/2aca0626-74aa-4e80-94cf-744eb75122b1}{Sakai}, under the appropriate assignment. One team will be randomly selected and their responses will be discussed on the screen. \\

In this activity you will read a short article and answer a few of question about a study. It is possible that the article doesn't provide the relevant information to answer certain question(s). If so, refer to the original study (linked). You do not need to read the entire paper, in fact, you are certainly not expected to understand all of it. Simply find the information that will help you answer the questions. \\

\textbf{Haters Are Gonna Hate, Study Confirms}
By  Katy Waldman \\
Published: August 28, 2013 on Slate, \url{http://slate.me/1dr5XPE} \\

Original study: Hepler, J., \& Albarracin, D. (2013). Attitudes without objects: Evidence for a dispositional attitude, its measurement, and its consequences. Journal of personality and social psychology, 104(6), 1060. \url{http://bit.ly/110XfNC}

$\:$ \\

\begin{enumerate}

\item What are the cases?

%\soln{Haters: 200 men and women}

\item What is (are) the response variable(s) in this study?

%\soln{Haters: Attitude towards the microwave oven}

\item What is (are) the explanatory variable(s) in this study?

%\soln{Haters: Whether the participant is a hater or not}

\item Does the study employ random sampling? How about random assignment?

%\soln{Haters: Via Amazon's MTurk - self selected sample, no random assignment.}

\item Is this an observational study or an experiment? Explain your reasoning.

%\soln{Haters: Observational, doesn't use random assignment.}

\item Can we establish a causal link between the explanatory and response variables?

%\soln{Haters: No}

\item Can the results of the study be generalized to the population at large?

%\soln{Haters: No}

\end{enumerate}



\end{document}